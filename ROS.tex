% Folien zur TAMS Selbstdarstellung
% 2009.02.24
% 2010.03.08 updates, fixes
%
\documentclass[t]{beamer}
%\usepackage{multimedia}		% for movies, sounds, animations...
%\usepackage[english]{babel}		% new english
%\usepackage[latin1]{inputenc}		% input...
\usepackage[utf8]{inputenc}		% SuSE 10.x default is UTF-8
\usepackage{tabularx}			% local, only for this doc.
%\usepackage[tams, engl, uniWZ, blockBG]{tamsBeamer}
\usepackage[ngerman]{babel}		% new german
\usepackage[uniWZ, tams, blockBG, engl, conference]{tamsBeamer}
%-----------------------------------------------------------------------------
%-- options		------------------------------------------------------
%			tams	|	- TAMS		publication
%			engl		- english strings	[german]
%			uniWZ	|	- uni		watermark
%			tamsWZ	|	- tams+uni	watermark
%			cinacsWZ	- cinacs+uni	watermark
%			secToc	|	- toc repetition at each section
%			secTocA		- -"-, all sections: show
%					  replacement for toc in short docs
%			subsecToc	- toc repetition at each subsection
%			secNum  	- (sub)-section numbering
%			fullstep	- always step through items
%			noFoot		- footline	off
%			noPage		- page numbers	off
%			noAuth		- author	off
%			conference	- footline with \foottitle{...}
%			blockBG	|	- block, example etc. background
%			blockRound	- -"-, rounded+shadow


% colors:     r g b 
% unihh red:  254 0 0
% dark red:   203 0 0
% light gray: 242 242 242
% dark gray:  122 122 120

% fonts definitions			--------------------------------------
% ----------------------------------------------------------------------------
% default: cmss, OT1 fontenc		good with UniHH font "The Sans"
%					-> don't change fonts!
%\usepackage{times}			% other fonts
%\usepackage[T1]{fontenc}		%

% document definitions			--------------------------------------
% ----------------------------------------------------------------------------
%\author[AutorA, AutorB]		% author	-- option: short
% {A.~Autor\inst{1} \and B.~Autor\inst{2}}%		-- option: \inst{...}
% style option: [tams] predefines institute...
% or define \institute{...}
%					% \inst{...} for different institutions
%\institute[Universities A and B]	% institution	-- option: short
%{ \inst{1}%
%  University of A\\
%  Department of A
%  \and
%  \inst{2}%
%  University of B\\
%  Department of B}


%%%%%%%%%%%%%%%%%%%%%%%%%%%%%%%%%%%%%%%%%%%%%%%%%%%%%%%%%%%%
%%%%%%%%%%%%%%%%%%%%%%%%%%%%%%%%%%%%%%%%%%%%%%%%%%%%%%%%%%%%
%%%%%%%%%%%%%%%%%%%%%%%%%%%%%%%%%%%%%%%%%%%%%%%%%%%%%%%%%%%%
\title[UHAM/T:Platform(s) and State of Simulation]{Platform(s) and state of simulation\\
  %Eine kurze Einführung in die Service-Robotik
  %Service-Robotik für alle?
  %{\small Die Perspektive der Service-Robotik}
}
\author[D.~Klimentjew, S.~Rockel, J.~Zhang]{{D. Klimentjew, S. Rockel, J. Zhang}\\ \small{klimentjew@informatik.uni-hamburg.de}}
%Prof. Dr. Jianwei Zhang\\
%Norman Hendrich, Denis Klimentjew
\institute{%
Universität Hamburg\\
MIN Fakultät, Fachbereich Informatik\\
Technische Aspekte Multimodaler Systeme\\
Vogt-Kölln-Str. 30, D-22527 Hamburg\\
\{klimentjew,rockel,zhang\}@informatik.uni-hamburg.de}
\date[2-december-2011]			% event/date	-- option: short
  {2.~December 2011}

\subject{TAMS, LaTeX, Folien}		% subject	-- option for pdf

\def\quelle#1{{\tiny \makebox(0,0){}\vfill\hfill #1}}
\def\ii{\item[]}


\begin{document}

\frame{\titlepage}


\begin{frame}
 \frametitle{\tocName}
 \tableofcontents
\end{frame}


%%%%%%%%%%%%%%%%%%%%%%%%%%%%%%%%%%%%%%%%%%%%%%%%%%%%%%%%%%%%
%%%%%%%%%%%%%%%%%%%%%%%%%%%%%%%%%%%%%%%%%%%%%%%%%%%%%%%%%%%%
%%%%%%%%%%%%%%%%%%%%%%%%%%%%%%%%%%%%%%%%%%%%%%%%%%%%%%%%%%%%
\section{Introduction}


\begin{frame}
\frametitle{Industrie-Roboter: Statistik}
%\picturegraphics{-8}{-55}{height=49mm}{images/kuka-regionen.png}%
%\picturegraphics{50}{-55}{height=49mm}{images/kuka-branchen.png}%
\vspace*{-5mm}
\strut
\begin{itemize}
\item weltweit ca. 1,1\,Million Roboter im Einsatz
\ii
\end{itemize}
\vspace*{45mm}
\quelle{(KuKA AG, 2008)}
\end{frame}


\begin{frame}
\frametitle{Service-Roboter: professioneller Einsatz}
%\includegraphics[width=99mm]{images/IFR-2009-service-robots.png}
\quelle{www.ifr.org/service-robots/statistics/}
\end{frame}


\section{ROS Stacks}

\begin{frame}
\frametitle{ROS Stacks}
%\framesubtitle{DESIRE, Rollin' Justin, PR2, AILA, Romeo, Omnirob}
\begin{itemize}
  \item Collect similar packages that work together
    \begin{itemize}
      \item ROS
      \item 2D Navigation
      \item Manipulation
      \item Controllers
    \end{itemize}
\end{itemize}
\end{frame}

\section{ROS Basics}

\begin{frame}
\frametitle{Basics}
\begin{itemize}
  \item ROS
  \begin{itemize}
    \item Meto operating system for robotics
    \item Obtain, build, write, and run code across multiple computers, and
    multiple robots
  \end{itemize}
\end{itemize}
\end{frame}

\begin{frame}
\frametitle{Basics High Level View}
\begin{itemize}
  \item ROS
  \begin{itemize}
    \item Simulation
    \item Task execution
    \item Mobile manipulation
    \item Navigation
    \item Visualization
    \item Client libraries
    \item Message passing
  \end{itemize}
\end{itemize}
\end{frame}

\begin{frame}
\frametitle{Basics}
\begin{itemize}
  \item Publisher sends Messages to Subscribers
  \begin{itemize}
    \item Usually TCP/IP transport
    \item XML-RPC is only used to negotiate transport (no messages via XML-RPC)
  \end{itemize}
  \item Uniquely named
\end{itemize}
\end{frame}

\begin{frame}
\frametitle{Basics}
\begin{itemize}
  \item Supported Platforms
  \begin{itemize}
    \item Linux, Mac OS, partial support for Windows
  \end{itemize}
  \item Languages
  \begin{itemize}
    \item C/C++, Python, Octave, Lisp, Java
  \end{itemize}

\end{itemize}
\end{frame}

\begin{frame}
\frametitle{Basics}
\begin{itemize}
  \item ROS File System Level
  %packages, stacks message, service
\end{itemize}
\end{frame}

\begin{frame}
\frametitle{Basics}
\begin{itemize}
  \item Nodes, Master, paramete server, topics, service, bags
\end{itemize}
\end{frame}

\section{Sensorics}
\section{Manipulation}
%Opening door video demo
\section{Navigation}
\section{Simulation}

\end{document}

